 1) Does NMSE work as a performance metric?
     Yes. Normalizing by the response mean or variance works fine and has little effect on the actual fit. 
     Results for 4 minute fits with boo12 (MSE) vs boo14(NMSE)/boo15/boo16 were quickly tested for batches 240 and 251. 
     boo12: MSE  = nanmean((p-r).^2)
     boo14: NMSE = nanmean((p-r).^2) / (nanmean(r)^2);
     boo15a: NMSE = nanmean((p-r).^2) / (nanmean(p)*nanmean(r));  (FAILS HARD)
     boo15b: NMSE = nanmean((p-r).^2) / (nanmean(p.*r));  (Didn't try it)
     boo16: NMSE = nanmean((p-r).^2) / (nanvar(r));
     TODO: Queue up some results

** Considerations when fitting per file:
   1. Normalization should NEVER be split, which means we can't just do naive "outer loop" methods that train on each file independently. 
   2. The strategy instead is to use "splitters" and add intelligence to the fitters so that it iterates 
   3. Must work for several performance metrics
   4. Safety check so that these modules do not allow unifiers/splitters: Correlation, MSE, bayesian, loader, and perhaps normalization?


* Notes on Variable step size boosting
  - The objective function given to boosting() doesn't provide the STACK or XXX.
    - I can hack around it and normalize by power of the difference, which is what stephen suggested
    - Or I can do the "pretty" thing and use the objective function
  - When doing "variable step size" boosting, you may not want to use step size as a termination condition.




TODO: MERGE WITH FITTER SURVEY. It appears that they do in some circumstances. This test fits the log compressor, the FIR filter, and the output nonlinearity (nonl, exp, sig, or zthresh) simultaneously (in the case of 'boost'), or iteratively for all other fitters. A keyword "fminlsqi" was created and tested on one or two but is pretty slow and doesn't appear to be significantly better than qlsqi. "Boost" is still the winner in terms of average performance, but in terms of ranking, 'boostis', which performs 100 iterations of a very short boost cycle, performs the best.


% \subsection{State-space model of cochlear vibration}

% There are three main parts, which for brevity will be referenced using the following mathematical symbols adopted from standard engineering control theory nomenclature:

%% \begin{eqnarray}
%%  \dot{x} & = A(x) + B(u)\\
%%  \dot{y} & = C(x) \\
%% \end{eqnarray}

%% where $x$ is the state of the system, $u$ is ....

%% \begin{description}[]
%%   \item [$A(s)\rightarrow u$] A \definition{filter} $p$ which transforms the time-indexed sound wave $x$ into stimulus intensity vector $u$. It models how the cochlea transforms the stimulus sound wave to the vibration of the basilar membrane. Vector $s$ has several components which may correspond to different filters. 
%%   \item [$q(u,x_{i-1})\rightarrow x_i$] A probably nonlinear \definition{activity function} $q$ which maps stimulus intensity $u$ and previous internal state $x_{i-1}$ to a new activity state $x_{i}$. 
%%   \item [$C(x) \rightarrow y$] A \definiton{spike noise function} describing the probabilistic nature of when spiking is observed. 
%%   \item [$m() \rightarrow z$] A \definiton{performance metric} describing 
%% \end{description}


%% \begin{qanda}

%% \Q How do we find both the model parameters and also the average rate of fire for the neuron? We know neither: our data just shows 'observed' firings from which we infer the true rate.
%% \A Treat both of them as unknowns. Use Expectation Maximization (EM) algorithm, a nonlinear model, and an Unscented Kalman Filter (UKF) to iterate between estimating the best model prameters and re-interpreting the data in light of those parameters until convergence occurs. % TODO: Insert blog diagram figure here.

%% \Q Pearson's r coefficient of correlation is a linear measure -- how can we use it to compare nonlinear systems?
%% \A Use feedback-linearization-style model inversion of the observation system, and we do the comparison in a linearized space. Let's use 'average rate of fire' since that is an easy place to start, and it neatly partitions the system model and observational model.

%% \Q How big should PSTH bins be? Small bins are very discontinuous and large bins smooth so much that significant information is lost. 
%% \A Trick question; let's avoid binning at all and instead use a method, such as a Kalman filter with a Poisson process observation model, which allows use of very small bins and interpolates the spiking rate between measurements, maximizing the information gleaned from the spiking data. PSTHs have other problems: there is no error bounds, they are just data. 

%% \Q Will the EM algorithm and Kalman filters cause phase lag in the model estimation?
%% \A I don't think so, but I've been wrong before.

%% \Q How should we include the preprocessing (like stimulus envelope discovery or gammatone filtering) in our framework?
%% \A Consider the preprocessing to be part of the model itself. That way, all variable parameters can be accomodated in the same fashion.

%% \Q What method will be used to optimize the values for a large number of parameters?
%% \A Several methods could be used: \begin{enumerate}
%%   \item Boosting
%%   \item Simulated annealing
%%   \item Conjugate gradient descent
%% \end{enumerate}

%% \Q What about evaluation of the performance? MUST be cross-validated for sanity.
%% \A \begin{enumerate}
%%   \item Simple correlation
%%   \item Jackknifed correlation (with error bars)
%%   \item N-fold cross validation error
%%   \item Error under L1, L2, Linfty norms
%%   \item Log-likelihood of the model (given some variance)
%%   \item Monte-carlo swarm average (checks log-likelihood of many many points)
%% \end{enumerate} 

%% \Q What is the to the optimization engine?
%% \A Optimization engines need
%%    ARGUMENTS:
%%        An initial parameter vector $\theta$
%%        An initial search variance $\sigma_{\theta}$ expressing some priors about the parameter vector. Setting a prior to variance 0 makes it act like a constant.
%%        A model function $M(\theta) \rightarrow {P, N, Q}$ that creates a model corresponding to those parameters $\theta$, where
%%           $P(\theta) \rightarrow G$ is the prefilter creation function 
%%           $N(\theta) \rightarrow Z$ is a nonlinear, state space model creation function
%%           $Q(\theta) \rightarrow \hat{Z}$ is the inverse observational model creation function
%%        An evaluation function $E(Z,\hat{Z}) \rightarrow r$ which measures the model's goodness of fit to the observed measurements
%%        Optimization state $\psi$, which is just a vector to store whatever relevant values the optimization function needs to make its decisions
%%        An optimization function $O(\theta, \psi, r) \rightarrow \theta$ which improves the estimate of the parameters based on the model performance
%%        A termination condition function $\Omega(\psi)$ which returns a boolean indicating that the optimization is complete. For example, a limit on the number of optimization iterations or a measurement of convergence. 

%% \Q What components must be defined for each model?
%% \A A model of a neuron has three parts ($P,N,Q$ as introduced above above): \begin{enumerate}

%%   \item[PREFILTER:] Turns raw waveform into a pseudo frequency-domain signal (may not be the true linear frequency domain, however). Models the cochlea and mechanotransduction.
%%       A. Gammatone filter (center frequency, bandwidth, phase locked or not, averaging window size)
%%       B. Square wavelet filter, ie DFT. (center frequency, bandwidth, averaging window size)
%%       C. Arbitrary FIR wavelet filter
%%       D. Possibly, a physical model-based representation such as a damped harmonic oscillator.
%%       E. Exotic wavelets: sinc, lanczos, mexican hat, meyer
%%       F. Envelope of the stimulus

%%       Prefilters can be created with functions $P(\theta) \rightarrow G$ which return filter functions $G(X,f_X) \rightarrow $. Functions $G$ have the following properties:

%%        ARGUMENTS:
%%            $X$    1xN vector of the stimulus
%%            $T_X$  Stimulus sample spacing interval [s]
        
%%        RETURNS:
%%            $Z$    1xM vector that is the filter's response
%%            $T_Z$  Filtered stimulus sampling interval [s]
   
%%    Prefilter creation functions $P$ accept model parameters $\theta$ that define precisely how the stimulus is filtered. If you wish to treat this as a hyperparameter, you may set it.

%%    \item [MODEL:] Turns that pseudo frequency domain signal into an average rate of neural activity. Models the neuron. 
%%       A. First order linear model with fixed delay
%%       B. Linear model with a finite impulse response
%%       C. Synaptic depression model. (Filter bank + weights)
%%       D. Multiple synapses with different latencies (Timing + weights)
%%       E. Self-referential ``Tire after firing'' stimulus model: increase threshold as stimulus increases (Refractory period)
%%       F. Log or sqrt compression at input or on output
      
%%    \item [SPIKING/OBSERVATIONAL INVERSE MODEL] Turns the discrete spike events into an estimate of the probabilistic process that generated them. 
%%       A. Poisson spiking with a smoothly varying average rate of fire (what I have now)
%%       B. No model, just binning spike events.
%%       \end{enumerate}

%% \Q What is the interface between data selection and model fitting?
%% \A A ragged vector of vectors of stimuli, X, a matrix of the responses to each stimuli Y, and the sample rates of each. Example:

%%    X = [stim1@t0, stim1@t1, ..., stim1@tn;
%%         stim2@t0, stim2@t2, ..., stim2@tn];
%%    Xfs = [0.01;  % 100Hz
%%          0.02]; % 50Hz
%%    Y = [resp1@t0, ...; 
%%      resp2@t1, ...]; 
%%    Yfs = [0.01; % Sample rate
%%           0.02];

%%    Requirements:
%%        All of the above must have the same number of rows
%%        cols(Y(i)) * Yfs(i) == cols(X(i)) * Xfs(i) must be true for any i

%%    This interface can therefore be used for fitting a model to a single data set, or even to multiple data sets. In the latter case, the data sets cannot be directly concatenated because that would create discontinuities in state that wouldn't fit the model. 

%% \Q What is the interface to the optimization system or systems?
%% \A The optimization function:
%%    ARGUMENTS: 
%%      params      An initial parameter vector $\theta$.
%%      mapfn       A function that returns a model function corresponding to parameters $\theta$. 
%%      modelfn     A function that returns a model function using parameters. 
     
%% \Q How can I map a large number of parameters into a model?
%% \A Define a function for each model that provides a mapping from a vector to a EM-runnable object.
%%    For example, 

%% \Q How can I visualize how well the model is doing?
%% \A Plot the scatter plots, correlation, and best fitting line. Also, plot the residuals!

%% \Q How can I visualize the effect of the parameters?
%% \A Plot the function defined by the parameters and histogram where the data falls on that function.

%% \Q How can I make a simple system for graphing results?
%% \A Flat files loadable by matlabe are probably the easiest. Each data file gets its own folder. Variables get names
%%    stim_fs.mat     Stimulus file at a particular frequency, channel 1
%%    resp_fs.mat     Response file at a particular frequency
%%    envelope.mat    The stimulus envelope at a particular frequnecy.
    
%% \Q How should I structure the model-estimation script to save results incrementally, be interruptable, and incrementally visible?
%% \A Basically, the main script will a file doesn't exist, then a function is called to create it. 
%%     setdatafile()    -> Sets the path to look at a particular directory.
%%     querydb()        -> original data files
%%     run_gammagram()  -> Creates a gramma spectrogram
%%     run_filter()     -> Created filter'd stimulus trace using the model (or envelope funciton)
%%     run_em()         -> Do expectation maximization algorithm (ON EVERY PART OF THE DATAFILE, WHICH MAY BE MULTIPLE EXPERIMENTS!)
%%     run_strf()       -> compute the STRF from convolving the gammagram and response data
%%     run_residuals()  -> compute the residuals
%%     run_bestfit()    -> compute best 2nd order linear fits for correlations of this model
%%     run_plotreport() -> Plots an all-encompassing report of the model

%%     Higher level functions:
%%     run_plotcorr()   -> Plot correlation scatter plots between various models. (to help you hunt for next nonlinearity). 
%%                         Plots signal histogram on diagonal, scatter plots on lower trinagle, residuals on upper triangle. 
%%                         Where to plot residual histograms or residual power spectra?

%% \Q How do I convert a function call with various parameters into a file name, and back again?
%% \A Define a function that converts that into a simple text string, and another that 
%%     run_fn(fn, params)     Returns a string
    
%% \Q What about higher level?
%%      Accepts a function to run
%%      Accepts some arguments
%%      Checks if arguments exist. If not, errors. 
%%      Converts 
%%      Checks if the function results already exist
%%      If not, checks arguments.
   
%% Function to generate new sample points in the unknown space
%% 1. Sample uniformly in the Nx2 vector between bottom a and top limits b
%% 2. Samples according to a gaussian with mean mu and covariance matrix Sig
%% 3.  

%% Checking assumptions are correct:
%%    1. Try a forward model structure with your initial starting model, getting an avg rate. 
%%    2. Plot the distribution of Inter-Spike Intervals, scaling time accordingly.
%%    3. Check the fit of various observational noise distributions against the ISIs
%%    4. Infer what the true state must have been in the context of that noise distribution.
%%    5. Fit model using that smoothed data and EM
%%    6. Do evaluation with the improved model

%% % Interesting algorithm could be: 
%% % Define limits on the space (center + gaussianity)
%% % Partition space into two halves
%% % sample randomly in each, evaluate the best point
%% % Pick the better of the two spaces as your new limits

%% % The gaussian k-means sampling algorithm (May not converge!)
%% % 1. Sample randomly at first, evaluate them.
%% % 2. Sort the samples tested by their scores (Create a matrix, use sortrows() to sort by first column)
%% % 3. Take the first 30 samples or so,
%% % 4. Work out the mean, variance on those top samples
%% % 5. Generate 1000 more samples with that mean and variance, evaluate them.
%% % 6. Loop to 2. 

%% % SCAAT Boosting algorithm
%% %  Possible problems: no guaranteed convergence for nonlinear systems. 
%% %      Randomizing the step size would help avoid loops of death
%% %      Allowing different step sizes for each vector element would help avoid scaling problems where one parameter is 0.001 and the other is 10000.
%% %      Having periodic random sampling of the space could also help
%% %      Having a second loop which samples many many points along the free vector until the optimum value is found, migth also help.

%% \end{qanda}

%%  Extend depression model to include presynaptic calcium concentrations (4 parameter depression/facilitation model)
%%           Presynaptic calcium concentration (and/or synaptic cleft neurotransmitter concentration), a function of presynaptic potential
%% 	  Calcium (and/or synaptic cleft neurotransmitter concentration) recovery rate
%% 	  Vesicle depletion rate as a function of presynaptic calcium concentration
%% 	  Vesicle recovery rate 
%%    3. [ ] Mechanotransduction potassium effects (depression model?)
%%    4. [ ] Nonlinearity of frequency sensitivity over A1 (and presumably middle-ear transfer fn)

%% \begin{itemize}
%% \item Replace the correlation coefficient with the jackknifed version.
%% \item mag007d-d1 	Pen 	12-03-13 	0.719+/-0.044* looks pretty good!
%% \end{itemize}

%% * How could functions be passed from one to the next?
%%   All functions are assumed to accept TWO arguments and return TWO arguments
%%   Those arguments are the \matlab{STACK} data structure and the X data structure

%% * When called with no arguments, each function returns 2 arguments:
%%   1. The 'meta' information about the function
%%   2. The 'params' struct, which specificies the default parameters

%% * What should I do about optimizing parameters for multiple models?
%%   Nothing at all, just yet! But I guess it could be done by iterating across the models as well.

%% * Besides optimization, it would be useful to be able to run the optimization-type thing to get plots which answer certain questions:
%%   Example: what is the effect on correlation as you vary the downsampling frequency, for each of these training groups? (Would be a nice graph)
%%   Example: Does using a certain preprocessing function improve or reduce the accuracy of most models' predictions?

%% * What is narf_modelpane used for?
%%   It was written:
%%   1. To visualize a model's behavior on data
%%   2. To edit a model easily and explicitly
%%   3. To encapsulate all assumptions about the model in a stack
%%   4. To select which parameters are to be optimized with an optimization routine. 
%%      (The optimization routine GETS a copy of the stack, plays around with the data, then SETS the stack again after optimization is complete.)

 
%% * MODULE FUNCTIONALITY
%% ** Preprocessing: Anything that creates the dat.().pp_stim field
%%    The big two filters are an elliptical bandpass and gammatone filters

%% ** Downsampling: Anything that creates the dat.().ds_stim and dat.().ds_stim_time fields.
%%    I decided that downsampling should only occur on the stimulus side, since the response already just be loaded at the frequency that you wish.
%%    If you are just doing simple correlation comparisons, you will want to downsample to the same frequency as your response. 
%%    The same if you are doing some sort of interpolated response comparison, but you will leave your response freq high, and apply a convolution over your response to 'smooth' it a bit.
%%    However, if you are doing ISI comparisons, you will NOT care about your response sampling frequency, and instead compute the ISI times. 
%%    To accomodate all theses cases, downsampling only works on the stimulus side.

%% ** FIR filters
%%    Your free paremeters are the number of coefficients in the filter, and how many filters you want.
%%    Each filter spans all of the input channels. (I think it makes more sense to have one filter which acts across all channels than many filters which only act on one channel each)

%% * Allowed Dimensions: How should can we accomodate the later addition of extra dimensions in the future, such as behavioral characteristics?
%%   Right now we have:
%%   In the future, we may have more. 
%%   The only way I can think about allowing multiple dimensions to vary arbitrarily would be to either:
%%   A) Somehow keep track of their numerical indexes as you go along, using a struct
%%   B) Avoid numerical indexes and use struct arrays or cell arrays everywhere? 
%%   Overall, option A sounds like the more efficient choice

%% * Tricky things:
%%   We may need to do an iteration procedure that treats one part of the model (IE, Linear FIR filters) differently from a nonlinear part (In my opinion, this is just a special case sampler)
%%   If you modify a function after starting up narf_gui, what will happen? (Right now, changes to the pretty-name and params will not be altered without restarting narf_gui, however if you fix the function itself then that is fine.)

%% * Possible refactoring
%%   1. Data ordering is perhaps nonstandard, since we need filter(B,A,X,[],2) instead of filter(B,A,X);
%%   2. Should PREFILTEREDSTIM be a 3D matrix, or is it more convenient to use as a mixture of cell array and 2H matrices.? 
%%      STIM [30x400000] (30 tones with 400000 samples in time each)
%%      RESP [30x400000x3] (3 reps)
%%      PREFILTEREDSTIM{numoffilters} and under each cell [30x400000]
%%   3.  Rewrite of dbchooserawfile() because it's so damn useful for selecting a file, but let's make it work for multiple stimulus files
%%       (Should also display well, site and have selectors for channel, unit, etc
%%   4. Use squeeze() to remove unneeded dimensions from a matrix.
%%   5. Why is it 'stimpath' and 'stimfile' but 'path' and 'respfile'. it should be 'resppath'?
%%   7. Why is loadspikeraster the only thing that cares about the 'options' struct?
%%   8. Where should the line be drawn between analysis in the DB, partitionining things for your search within the DB, holding out data, etc?


  1. [X] How can sane initial conditions for optimization be automatically arrived at without extra script-writing?
	 Auto-initialization of model params is done by allowing modules to update their design based on the data by calling the optional 'auto_init' method.
	 Arg 1 is the STACK, not including the model itself. 
	 Arg 2 is the XXX data input, not including the model's output data itself. 
  2. [X] How can jack-knifing be integrated in to the optimization routine to prevent over-fitting?
	 Split the big long RESP and STIM vectors in fit_with_lsqcurvefit into 10 chunks
	 Take groups of 9 of those chunks, run lsqcurvefit, then test on remaining chunk
	 Take weighted average of all jackknifed solutions, weighting each by inverse variance? Or just mean, if we assume they all have same variance?
	 Return weighted average.
  3. [X] How should optimization constraints be incorporated in the design?
	 Probably the easiest way is to define a structure which may be used by pack/unpack to create upper and lower bounds, which are then passed to the optimization routine
	 opt_hints = struct('alpha', [-1 3], 'beta', [0 inf]); % Constrain alpha from -1 to 3 and beta from 0 to infinity. 
  4. [X] How should models be automatically generated in a quick and scriptable way?
	 See analysis/test_likely_candidates.m
  5. [X] How can design internal degrees of freedom be detected and corrected during optimization?
	 (Probably they cannot!)
  6. [X] There needs to be a place to store information about a whole model. 
	 For example, 'model name' and 'fitter' are two examples of fields that don't really belong in a module.
  7. [X] There is no best fitting routine, only fitting routines which work better for different cells. Allow them all a chance to run by making them module parameters.
  8. [X] Can jackknifing or the equivalent be applied to ANY fitting routine as a higher level function
	 If we only have one data file, how can we hold out some fraction of the stimuli so that we can do training/test on a single data file?
	 Solution:
	 - Fit routines use a 'score'
	 - The stack gives the score
	 - The score needs to be calculated from a jackknife
	 - How can data be jackknifed without modifying the stack?
	 - Immediately after the loading, zero a chunk of the stim and respavg (save the original, of course)
	 - Do a fit with whatever routine you want
  9. [X] N-step fitter (train FIR in common, train NL across each separately)
	 Surprisingly difficult to make several models need to be fit all on the same data. yet ALSO need to run on different behavioral states. 
         1. Violates my implicit expectation of 1 fitter -> 1 model. Now I have 1 fitter-> many models.
	 2. Now that training_set{} may be edited, it shouldn't really be copied from one XXX{1} to XXX{2} and so on.
	 Solution ideas: 
	 - Quick hack: five new fitters added
	   NL1, trains on all, but only trains NL on 1st
	   NL2, trains FIR on all, but only trains NL on 
  4. [X] Nonparametric Nonlinearity (NPNL) linearizes anything. 
	 It is very much data-driven, which is great. 
	 On the other hand, it fits itself to linearize almost anything, so we somehow learn less than a simple, parameter-driven model. 
	 How can we balance complexity in the FIR or complexity in the NL?
	 ANSWER: Sparseness needs to be modeled on the FIR side, Smoothness on the NL side. 
  5. [X] How can LSQ curve fit use sparseness and smoothness metrics?
	 You can cheat and destroy the module system by looking later in the STACK for the MSE element. 
	 If the MSE module exists and has nonzero weights, add a bogus zero element the LSQ target vector, and a bogus LSQ prediction vector element with a value of the sqrt(smoothness_penalty).	  

* Notes on Hyrax

1. If the scheduler crashes (indicated by new jobs not starting and or a last tick warning at the bottom of the queue status screen):
  log in to hyrax
  run matlab
  >> baphy_set_path
  >> narf_set_path    % assuming you've pulled the latest narf as of this evening
  >> dbqueuemaster
  enter your password when prompted to permit sudo. 
That should take care of everything.   It just occurred to me that it might be easiest in general to run the scheduler in a window on the actual hyrax machine so that anyone interested can take a look to see what it's up to.

2. Monitor queue from home via reverse tunnel through umd.edu:
  - open firewall back door by surfing here:  http://kamzik.org/rr22307.php
  - enter user "lbhb" password "feartheferret"  This will open the back door to the IP of your current computer for a month.
  - surf to http://bhangra.isr.umd.edu:19888   -- this is equivalent to http://hyrax.ohsu.edu/
Hi Stephen,

The data from batches 240 and 242 are in. Unfortunately, batches 233 and 241 had problems and haven't been run through fully yet. First, I noted a few weird things in the batch files, which may be my fault:

1. In batch 240, the following cellid's had identical training and test sets: por023a-c2, por031a-09-1, por31a-19-1
2. In batch 240, relative to the data I took last time, por028d-b1 was removed (despite it being an excellent cell), and por028d-c1 was not removed (despite it being a crap cell). I think that was probably done backwards.
3. In batch 242, the following cellid's had identical training and test sets: por025a-c1, por025a-d1, por026a-d1, por026b-a2, por026b-b2, por026b-c2.

In this experiment, I varied:
- Including a prestim silence or not.
- Adding a penalty for smooth FIR coefficients or not. 
- Using the nonparametric nonlinearity or a sparse empirical nonlinearity (essentially a gaussian mixture model)
- Using boosting or a fminsearch + lsqcurvefit, the best two search methods on average so far. 

The questions I was trying to answer were:

1. Does the prestim silence improve the average quality of the fits generated?   
   Yes. Only 9 of the 38 neurons had fits that were improved by removing the prestim silence. Interestingly, every single time the fit was improved by removing the prestim silence, the nonlinear saturation was low, as the following data table shows. Rows are sorted by sorted by nonlinearity saturation level. 
     | CELLID       | Init? | COMP  | NL   | BEST FITTER | R^2 | STRF      | Saturation | NOTE            |
     |--------------+-------+-------+------+-------------+-----+-----------+------------+-----------------|
     | por028d-d1   | yes   | l2    | npnl | fminlsq     | .24 | 1-Crap    | High       |                 |
     | por025a-c1   | yes   | log2b | npnl | boost       | .32 | 5-Perfect | High       |                 |
     | por028b-d1   | yes   | log2b | senl | boost       | .33 | 5-Perfect | High       |                 |
     | por031a-09-1 | yes   | l2    | npnl | fminlsq     | .06 | 2-Noisy   | Linear     |                 |
     | por026c-c1   | yes   | log2b | npnl | boost       | .52 | 5-Perfect | Linear     |                 |
     | por023a-b1   | yes   | log2b | npnl | boost       | .56 | 5-Perfect | Linear     |                 |
     | por026a-d1   | yes   | log2b | npnl | boost       | .15 | 3-OK      | Linear     | Late resp.      |
     | por027a-b1   | yes   | log2b | senl | fminlsq     | .28 | 3-OK      | Linear     | Late resp.      |
     | por027b-b1   | yes   | l2    | npnl | boost       | .21 | 3-OK      | Linear     |                 |
     | por025a-c2   | no    | log2b | senl | fminlsq     | .10 | 1-Crap    | Low        |                 |
     | por026a-b1   | yes   | log2b | senl | fminlsq     | .36 | 2-Noisy   | Low        |                 |
     | por026c-a1   | no    | l2    | senl | fminlsq     | .35 | 3-OK      | Low        |                 |
     | por026b-b2   | yes   | log2b | senl | fminlsq     | .53 | 3-OK      | Low        |                 |
     | por026c-d2   | yes   | l2    | npnl | boost       | .11 | 4-Clean   | Low        |                 |
     | por026b-b1   | yes   | log2b | senl | fminlsq     | .38 | 4-Clean   | Low        |                 |
     | por025a-d1   | yes   | log2b | npnl | fminlsq     | .50 | 4-Clean   | Low        |                 |
     | por025a-b1   | yes   | log2b | senl | fminlsq     | .54 | 4-Clean   | Low        |                 |
     | por024a-b1   | yes   | log2b | npnl | fminlsq     | .47 | 5-Perfect | Low        |                 |
     | por023a-c2   | no    | l2    | senl | boost       | .16 | 4-Clean   | Low        | Late resp.      |
     | por027b-b1   | yes   | l2    | npnl | boost       | .17 | 4-Clean   | Low        | Late resp.      |
     | por028d-a2   | no    | log2b | npnl | boost       | .21 | 4-Clean   | Low        | Late resp.      |
     | por026b-a2   | no    | log2b | senl | boost       | .30 | 4-Clean   | Low        | Late resp.      |
     | por026b-d1   | no    | log2b | senl | boost       | .28 | 5-Perfect | Low        | Late resp.      |
     | por024b-b1   | no    | l2    | senl | boost       | .30 | 2-Noisy   | Low        |                 |
     | por026c-a1   | yes   | log2b | senl | boost       | .67 | 5-Perfect | Low        | Exceptional.    |
     | por028d-a2   | no    | l2    | npnl | fminlsq     | .32 | 2-Noisy   | Low        |                 |
     | por028b-d1   | yes   | l2    | npnl | boost       | .36 | 3-OK      | Notch      | Late resp.      |
     | por028d-d1   | yes   | log2b | npnl | boost       | .32 | 3-Clean   | Notch      |                 |
     | por024a-c1   | no    | l2    | senl | fminlsq     | .12 | 2-Noisy   | Sigmoid    |                 |
     | por024a-a1   | yes   | log2b | npnl | fminlsq     | .31 | 3-OK      | Sigmoid    |                 |
     | por027a-a1   | yes   | l2    | npnl | fminlsq     | .13 | 4-Clean   | Sigmoid    |                 |
     | por028b-c1   | yes   | log2b | senl | boost       | .21 | 5-Perfect | Sigmoid    |                 |
     | por026b-a1   | yes   | l2    | senl | boost       | .56 | 5-Perfect | Sigmoid    |                 |
     | por026b-c1   | yes   | log2b | npnl | boost       | .20 | 4-Clean   | Sigmoid    | Late resp.      |
     | por024b-c1   | yes   | l2    | npnl | fminlsq     | .49 | 4-Clean   | Sigmoid    | Late resp.      |
     | por026c-d1   | yes   | l2    | senl | boost       | .18 | 4-Clean   | Sigmoid    | Unusually late. |
     | por026c-b1   | yes   | log2b | npnl | fminlsq     | .34 | 3-OK      | U          |                 |
     | por028b-b1   | yes   | log2b | senl | boost       | .19 | 4-Clean   | U          |                 |
 
2. Does smoothness seem to help our fits, on average?
   Only in certain cases. Although 15/38 cells had best models which had been smoothed, only 5 of those smoothed results were clearly better than the best unsmoothed model.
   The method by which I tested this was to smooth the average response with FIR filter with coefficients: [1 4 1]. This is a weak low-pass filter or smoothing kernel. It was only applied to the training set, so that smooth FIR coeficients would naturally occur. The training set was left untouched, so that it would still be a fair comparison with non-smoothed results. 
   The following table, sorted by R^2 value, shows that smooth results tended to help more for the poorly correlating results, presumably because it reduced overfitting. Although I expected it to also be more important to smooth coefficients found via FMINLSQ than boosting, this doesn't appear to be the case. Boosting can apparently also generate fits which benefit from some mild smoothing.
     | CELLID       | Init? | COMP  | NL   | BEST FITTER | R^2 | STRF      | Saturation | NOTE            |
     |--------------+-------+-------+------+-------------+-----+-----------+------------+-----------------|
     | por026c-a1   | yes   | log2b | senl | boost       | .67 | 5-Perfect | Low        | Exceptional.    |
     | por026b-a1   | yes   | l2    | senl | boost       | .56 | 5-Perfect | Sigmoid    |                 |
     | por023a-b1   | yes   | log2b | npnl | boost       | .56 | 5-Perfect | Linear     |                 |
     | por025a-b1   | yes   | log2b | senl | fminlsq     | .54 | 4-Clean   | Low        |                 |
     | por026b-b2   | yes   | log2b | senl | fminlsq     | .53 | 3-OK      | Low        |                 |
     | por026c-c1   | yes   | log2b | npnl | boost       | .52 | 5-Perfect | Linear     |                 |
     | por025a-d1   | yes   | log2b | npnl | fminlsq     | .50 | 4-Clean   | Low        |                 |
     | por024b-c1   | yes   | l2    | npnl | fminlsq     | .49 | 4-Clean   | Sigmoid    | Late resp.      |
     | por024a-b1   | yes   | log2b | npnl | fminlsq     | .47 | 5-Perfect | Low        |                 |
     | por026b-b1   | yes   | log2b | senl | fminlsq     | .38 | 4-Clean   | Low        |                 |
     | por028b-d1   | yes   | l2    | npnl | boost       | .36 | 3-OK      | Notch      | Late resp.      |
     | por026a-b1   | yes   | log2b | senl | fminlsq     | .36 | 2-Noisy   | Low        |                 |
     | por026c-a1   | no    | l2    | senl | fminlsq     | .35 | 3-OK      | Low        |                 |
     | por026c-b1   | yes   | log2b | npnl | fminlsq     | .34 | 3-OK      | U          |                 |
     | por028b-d1   | yes   | log2b | senl | boost       | .33 | 5-Perfect | High       |                 |
     | por028d-a2   | no    | l2    | npnl | fminlsq     | .32 | 2-Noisy   | Low        |                 |
     | por025a-c1   | yes   | log2b | npnl | boost       | .32 | 5-Perfect | High       |                 |
     | por028d-d1   | yes   | log2b | npnl | boost       | .32 | 3-Clean   | Notch      |                 |
     | por024a-a1   | yes   | log2b | npnl | fminlsq     | .31 | 3-OK      | Sigmoid    |                 |
     | por024b-b1   | no    | l2    | senl | boost       | .30 | 2-Noisy   | Low        |                 |
     | por026b-a2   | no    | log2b | senl | boost       | .30 | 4-Clean   | Low        | Late resp.      |
     | por026b-d1   | no    | log2b | senl | boost       | .28 | 5-Perfect | Low        | Late resp.      |
     | por027a-b1   | yes   | log2b | senl | fminlsq     | .28 | 3-OK      | Linear     | Late resp.      |
     | por028d-d1   | yes   | l2    | npnl | fminlsq     | .24 | 1-Crap    | High       |                 |
     | por027b-b1   | yes   | l2    | npnl | boost       | .21 | 3-OK      | Linear     |                 |
     | por028d-a2   | no    | log2b | npnl | boost       | .21 | 4-Clean   | Low        | Late resp.      |
     | por028b-c1   | yes   | log2b | senl | boost       | .21 | 5-Perfect | Sigmoid    |                 |
     | por026b-c1   | yes   | log2b | npnl | boost       | .20 | 4-Clean   | Sigmoid    | Late resp.      |
     | por028b-b1   | yes   | log2b | senl | boost       | .19 | 4-Clean   | U          |                 |
     | por026c-d1   | yes   | l2    | senl | boost       | .18 | 4-Clean   | Sigmoid    | Unusually late. |
     | por027b-b1   | yes   | l2    | npnl | boost       | .17 | 4-Clean   | Low        | Late resp.      |
     | por023a-c2   | no    | l2    | senl | boost       | .16 | 4-Clean   | Low        | Late resp.      |
     | por026a-d1   | yes   | log2b | npnl | boost       | .15 | 3-OK      | Linear     | Late resp.      |
     | por027a-a1   | yes   | l2    | npnl | fminlsq     | .13 | 4-Clean   | Sigmoid    |                 |
     | por024a-c1   | no    | l2    | senl | fminlsq     | .12 | 2-Noisy   | Sigmoid    |                 |
     | por026c-d2   | yes   | l2    | npnl | boost       | .11 | 4-Clean   | Low        |                 |
     | por025a-c2   | no    | log2b | senl | fminlsq     | .10 | 1-Crap    | Low        |                 |
     | por031a-09-1 | yes   | l2    | npnl | fminlsq     | .06 | 2-Noisy   | Linear     |                 |

3. Does the sparse nonlinearity work better than the nonparametric nonlinearity?
   In some cases, yes. The key difference, when there is one, seems to be in how well the nonlinearity extrapolates beyond the bounds of the training set data. 
   A little more work here could really pay big dividends, I believe. 

4. Which is more important, sparsity or smoothness?
   Sparsity is probably more fundamental and could probably achieve many of the same effects as smoothness. But that's just intuition and I haven't quantified it yet.   

Unless you have something pressing or an intuition you want to try, I'm going to concentrate on finding better nonparametric nonlinearities, especially in regards to how to extrapolate better beyond the edges of our data. That can have a pretty big effect on FIR coefficient cleanliness, as well as the correlation performance. 

After that is nailed down, I think the next approach would be to concentrate only on sparse fitters and stronger shrinkage methods. If we solidify the model structure as having a parameter-free compressor and nonparametric nonlinearity, we can just concentrate on the STRF and start running experiments to see just how sparse the FIR filter can become before performance significantly degrades. Studying this too much before the nonlinearity is better optimized could yield premature conclusions.

One other thought I had: should we also be considering a nonparametric compressor of some sort, or none at all? A log compressor makes sense biologically, but does it have any real effect given the nonparametric nonlinearity's ability to account for nearly any smooth nonlinearity, mathematically speaking?


Ivar
  7. [X] Why doesn't SB fitter seem to respect smoothness penalties?
	 Because it steps in the direction which best describes the data, which may have nothing to do with sparsity

  5. [ ] Replace 'model groups' abstraction with a list of arbitrary nested model-perturbing functions and associated keywords
	 Why: In the future, more information should be stored in the META structure:
         META.fitter = ...;  % Fitting tools are not specific to a single module, but are actually more global
	 META.scorer = ...;  % Performance metrics are likewise not really modules
	 Why: It would allow mutation of multiple parts of the stack, simultaneously. 
	 Why: Fitters need to have their own arbitrary string to describe their actions, and this is OUTSIDE of the normal module keyword system
  8. [ ] Fitters need to be composable
	 1. Fit the FIR coefs with reverse correlation first
	 2. Then fit the FIR coefs with Boosting
	 3. Then fit the NL part with fminlsq
	 4. Then have a loop where you do one boosting step and one lsq step.
* Discussion with Stephen

** Solved problem: How to initialize complex models in complex ways
   No other way around it; defining a new keyword function is the only truly flexible way to do this. 
   Some complex keywords can even add and initialize a model in the same step.
   Our never-ending task is to isolate the functionality of each keyword in the most logical and recombinable manner.
   But that's not a hard problem, since we can always refactor functionality we find ourselves cut-and-pasting.

** Unsolved problem: How should we accomodate our need to be able to study a model
   Although right now we are just looking at FIR and NPNL and how they vary when trained on one/multiple files.
   We can't just train five models, because we want the REST of the model to use all the same data.
   We can anticipate that we will be doing this all/individual separation of data for almost every computational module.
   
*** Option 1: Use wiring to connect many simple modules which use inputs that are all- or per- file
    Pros: 
    + Using 'wiring' to solve the problem can be done algorithmically, and doesn't change the modules.
    + A module which 'splits' a signal by respfile channel 
    + No change to existing modules
    Cons:
    - Makes comparison graphing more difficult
    - Confuses the idea of a 'model structure' with the 'experimental structure' 
    - Increased number of modules
    - Tree navigation required to some extent.
    - Looks like a bookkeeping nightmare having to name all those signals programmatically

*** Option 2: Make the all-/per- file analysis local to each module.
    Pros:
    + Fewer modules are created, improving fitting/plotting speed
    + The all-/per-file comparison is so much easier because the data is local
    + No changes to gui code
    + You can graph all or per simultaneously
    Cons:
    - Changes to EVERY SINGLE module that wants to implement this.      
    - Data structures become more complex

*** Discussion
    It has to be Option 2: change every single module to accomodate this all/per file experimental thing. 
    This is the only option because it would be very bad to confuse the experimental structure with the model structure. 
    Our experiment may bei asking how a module's parameters look in various behavior states, but that's not the same thing as having branches in a model.
    We also have a real need to train part of the model on everything, and another part only on specific respfiles.
    Keeping track of if we are looking at the Inhibition or Excitation FIR filter when you have 8 Respfiles would be too confusing...that's 16 FIR filters, plus splitters/joiners!

*** Proposed Solution:
    What we really need is a uniform abstraction that can be applied to any module.

    1. Manage module fields for both all-file or per-file cases
    2. Manage module plotting for all- or per- cases
    3. Align training/test set pairs when using a particular parameter set grouping
    4. Remains customizable in the event that we need to create different groupings (say, active1 vs active2 vs passive behaviors) 

    The way to achieve this uniform abstraction is to change the functional interface of each module:

    1. Every module function interface now accepts:
       - the module parameter set to use
       - the respfiles/data to act upon
       - previous STACK
       - previous XXX

    2. Modules now have an optional flag
       - When true, arguments to functions are given
       - The respfile cell array is specified by some 'experimental respfile selector function'

    3. The default function of a modules no longer works on ALL data, but only on the respfile argument.

    4. Plot functions of modules no longer works on ALL data, but only on the respfile cell array argument.
       Modules no longer plot directly to their output
       Instead, they accept an argument of where to plot, 
       They also receive the module fields to use when plotting

** PROBLEM: Each model structure will have a different way of defining sparsity
     Some models have 1 FIR filter, some have per-file filters, some have IN/EX filters
  Therefore, don't build it in to the MSE module...it is instead part of a fitter.
  
* SPLIT PARAMETER SETS
 SUMMARY OF CHANGES:
  1) [ ] Jackknifing is a generic FITTER function keyword and requires no special work.
         - Besides, we need to do jackknifing even on split models.
  2) [ ] Peeking at another module's data without using EXTRACT_MODULE_PARAMS() is now STRICTLY FORBIDDEN
         - You provide the index number, which you found previously with find_modules()
  3) [ ] STACK is now a 2x deep cell array. XXX has no change.
  4) [ ] Modules all have a 'splitter' fn and a 'unifier' fn
	 - The splitter function is called when append_module() is done
	 - The splitter accounts for data conditions
	   {[mdl1, dat1], [mdl2, dat2]} = splitter(STACK, XXX)
	 - The unifier brings back multiple data conditions
	   XXX{N+1} = unifier(x1{n+1), x2{n+1});
  5) [ ] recalc_XXX now 
	   1. calls mdl.fn() on each split parameter set
	   2. calls mdl.unify() on the XXX results of all of those
  6) [ ] Whatever calls plot hooks in narf_modelpane now
	   1. calls splitter() to get the group parameter sets
	   2. Creates a pack of axes()
	   3. Calls the selected plot function on each one
  7) [ ] Pack and unpack_fittables now understand split parameters

* Should I put selected GUI info  in the XXX data struct?
    Pros:
    + No extra argument    
    Cons:
    - Confuses GUI state and XXX data
    - Need to go through XXX data  
    - Could accidentally overwrite XXX data
    BAD IDEA!
* Module Interface Changes
  - DO functions now get passed four arguments instead of two:
    | mdl   | Module parameter set                                                   |
    | x     | default xxx data to work on                                            |
    | stack | The whole stack. It's STRONGLY ENCOURAGED you don't actually use this. |
    | xxx   | The whole xxx. It's STRONGLY ENCOURAGED you not use this.              |

  - PLOT functions now get passed six arguments
    | selected | Struct containing plot_gui selections                            |
    | stack    | The whole stack up to this point, including alternate param sets |
    | xxx      | The whole xxx up to N+1 with the unified results                 |

  - These can easily be computed from stack and xxx
    | mdls   | cell array of module parameter sets for this module |
    | xprevs | cell array of split inputs                          |
    | xnexts | cell array of outputs before the unify              |
 
  - Selected struct has these fields
    | selected.stimfile    | Selected stimfile              |
    | selected.stim_idx    | Selected stim idx              |
    | selected.chan_idx    | Default selected channel idx   |
    | selected.parmset_idx | Default selected parameter set (or this parmset) |

  - The selected struct is built by walking the NARFGUI array and merging fields

  - When a plot-gui is changed, it triggers a REFRESH_GUI from that point
    Model the logic on the recalc_xxx and it should be fine

  - Only remaining questions is should I
    1. Plot per parameter set and encapsulate loops?
    2. Plot entire thing

    I guess it makes more sense to plot the whole damn thing
    If I just want to plot part of it, I do: 

  - For encapsulation at some point, we may want to remove global defaults:
    1. We should pass a handle to the plot axes to the plot functions

* Meta-Methods
** I need a generic plotting method which takes
  - A function that plots a single line
  - x axis vector-valued cell array {x1, x2, x3}
  - y axis matrix-valued cell array {Y1, Y2, Y3}
  - fieldnames (Assumed to be parameter set names)
  - channames (assumed to be channels and corresponding with Y's 2nd dimension)

** It should also:
   - Colors indicate channel: b, g, r, c, m, 
   - Line types:    -, --, -., :, o-, x-,
   - Automatic legend, X axis, Y axis, etc
   - plot(dat.(mdl.output_stim_time), ...
       dat.(mdl.output_stim)(:, sel.stim_idx, sel.chan_idx), 'k-');

* Have the FIR module be MIMO, and then SUM the outputs
  Is it useful to do this to simplify plot routines? Or not?
  99% of the time I want the summing, so let's include it. 

* Bayesian perspective
  You have a model structure, which is the TYPE of module in the (STACK)
  You have several model parameter sets (Sideways sets of parameters in STACK)
  Each of those has a likelihood/data chain (XXX)
  Now that we can have multiple parameter sets, we can even have ENSEMBLE models
  (Use a gaussian random splitter, and a mean unifier or one that discards outliers)
  
A true bayesian would allow parameter sets to vary across the WHOLE stack.
There's no reason we couldn't do this with a tree-shaped STACK.
It would be like encompassing multiple models in the same file, all working on the same data.
The downside would be that aggregating data for plotting would be a bitch.
Training would also be a bitch. 
For now, being able to localize changes to a single module is good enough. 
Besides, we already have enough rope to hang ourselves with.

* Why the plot functions have to change so much:
  Because right now in EVERY single plot function, we check the status of a GUI that may not exist
  For better flexibility, we should allow plotting to occur programmatically
  This can only occur if plot functions are GIVEN the GUI status as an argument
* SUMMARY OF CHANGES:
  1) [X] Jackknifing is a generic FITTER function keyword and requires no special work.
         - Besides, we need to do jackknifing even on split models.
  2) [X] STACK is now a 2x deep cell array. XXX has no change.
  3) [X] Modules all have a 'splitter' fn and a 'unifier' fn
	 - The splitter function is called when append_module() is done
	 - The splitter accounts for data conditions
	   {[mdl1, dat1], [mdl2, dat2]} = splitter(STACK, XXX)
	 - The unifier brings back multiple data conditions
	   XXX{N+1} = unifier(x1{n+1), x2{n+1});
  4) [X] Pack and unpack_fittables now understand split parameters
  5) [X] Peeking at another module's data without using EXTRACT_MODULE_PARAMS() is now STRICTLY FORBIDDEN
         - You provide the index number, which you found previously with find_modules()
	 - {cellarrayofparams} = Extract_module_params(module_index, param) 
  6) [X] recalc_XXX now 
	   1. calls mdl.fn() on each split parameter set
	   2. calls mdl.unifier() on the XXX results of all of those
  7) [X] Whatever calls plot hooks in narf_modelpane now
  8) [X] Substantial work in utils/
	 All the functions which use STACK should now be more or less workable!	  
  9) [X] There are three splitter implementations available:
	 1. per-respfile
	 2. per-filecode category
	 3. all-files (No splitter provided)
  10) [X] There is one generic unifier
          1. Per-respfile merger
  11) [X] STACK{}.gh is now its own structure: NARFGUI
	  - [X] Ensure no more *.gh things exist
	  - [X] Ensure that all the plot_gui's now are in NARFGUI
	  - [X] get_baphy_plot_controls should become irrelevant now
  12) [-] Substantial work in pretty much every module which touches STACK naively
	  - [X] load-stim-from-baphy
	  - [X] normalizer
	  - [X] FIR
	  - [X] LOG2b
	  - [X] NPNL
	  - [ ] NPFNL
	  - [ ] Other?
Hi Stephen, 

I just thought I would update you on NARFv2. It's pretty much like NARFv1 but with some programmatic interface changes that streamline the modules a bit better and let us do a few things we couldn't do before:

1. Plotting via scripts is now far more controllable. This was not possible at all before, strangely enough, since I had focused on plotting from the GUI.

2. Each module can have multiple 'parameter sets' of data. In other words, the FIR module can now have multiple sets of coefficients which work on different parts of the data, or even on the same data if you want. The function which splits up the data and passes it to each parameter set is called a 'splitter', and the one which brings the data back together again is called the 'unifier'. I intended this to solve our problem of needing to do per-respfile or per-behavioral condition fits for just part of the model. However, this could be interesting for doing something like 'consensus' or 'ensemble' modules, which each could be trained on some jackknifed subset of the data. We could then throw out the worst FIR coefs (which were sensitive to outliers), and average the rest of the values to make a prediction. 

3. Arbitrary initialization of any part of the model, at any step in the fitting process, is possible with a keyword system. If you want to do two-step fitting routines, this should become much simpler now. 

4. Modules are smart enough to plot their multiple parameter sets in the same plot window for easy viewing (i.e., multiple NPNLs can be shown on the same axes plot, or multiple FIR filters can be shown on the same FIR coefs plot).

5. Keywords can be combined in a clever recursive fashion to allow us to test combinationss in an extremely complex way. We now have plenty of rope to hang ourselves with.

6. We can go into the details of the changes; it is slightly more painful to mess with STACK manually now, but on the other hand the added functionality is probably worth it. This isn't a huge problem -- you should be using "append_module" to build a model whenever possible anyway. 

7. Plotting routines are standardized and work for multiple parameter sets and multiple channels. I added intelligent legends and labels for the NARF newbies as well.

The only remaining things to do before I push the code back in are:

1. Fixing some awkwardness with plotting multiple parameter sets in the narf_modelpane gui

2. Finishing testing the splitter/unifier code for the minimum relevant modules (FIR, NPNL, NPFNL)

3. Testing that the boosting fitter still works as intended (probably fine).

4. Check that I didn't break the job system in anyway (probably fine).

I'd guess it will take me another 2 days to finish those, but it may be more or less, depending on how smart I am feeling. 

I'll be here next week on Monday and Tuesday. Whether I can come in this Friday depends on how much I get done Wed/Thu of this week. 


Ivar


* Talking Points for Stephen
** Multiple paramsets
  - I'm allowing load-stim-from-baphy to have them too
    - Why: Perhaps someday we will want to integrate data from MULTIPLE cellids
    - We can do this by hacking out a strange 'splitter & unifier' for loadstimfrombaphy
    - Allowing such aggregation of data doesn't increase baphy's or narf's complexity
** Recalculating
  - There is more 'recalcing' going on than before during plotting
  - It sounds goofy, but plot functions now recalc all the data, instead of just calling plot functions 
** Plotting
   - Right now there's no GUI selection for module paramset, but there will be soon

* Questions for Stephen
  - I notice that more jacknifes (10 instead of 5) will mean more jackknifes will share data.
    As we have more and more jackknifes, they will converge on being almost exactly the same, except for 1 data point's effect
    Therefore, predictions will also tend to be pretty much identical as if we hadn't jackknifed at all?
    However, if we only have 2-4 jackknifes, they share no data
    The prediction in one may be pretty different from the prediction in the other. 
    That seems closer to what we are looking for: if each predicts the other poorly, we are clearly not sparse enough.
    Maybe that's still true for more jackknifes, but it just escapes my intuition?

* Options
** Overwrite the test_set file with the held-out data, and use correlation's test_set result to pick the winner
*** Strategy:    
    1. Fitter removes the test-set data and replaces it with the held-out data (Which is NAN every where else)
    2. Complete duplication of the STACK in a local cache so that we can pull out coefs at the end
    3. Each jackknife generates a prediction
       1) These predictions are stored in an XXX struct, since we DO need to keep them around, if only to aggregate the results
       2) The predictions are made by overwriting the test set.
    4. Each jackknife's predictions are merged. Specifically, meaning that:    
       1) Predictions must be stored in a special array of XXX structures, one for each jackknife
       2) Every element of XXX from 2 onward needs to have merging occur IFF it is a fittable_field one
       3) The non-NAN elements are merged
    5. Computing the MSE is going to be a bitch, because:
       1) If I 'calc_xxx', I lose the merged data in one swell foop!
       2) Do I propogate everything down until it reaches the MSE thing? Or what?
       3) The MSE needs to work on the test set data. 
    6. The test-set performance is cached in many XXX, so aggregated data cannot be fed through modules in the normal way
       We _could_ move the MSE into META, since it's part of the performance metric, after all.
       BUT, then there's no good place to view it in the GUI, and I want to allow user-defined modules
       1) I could mark performance metric modules somehow and feed them the data, then recompute
       2) I could make sparsemse module
    7. Then, the fitter restores the test set data when done
*** Problems:
    1. It's VERY LIKELY it interacts poorly with the splitter functions, since they use test set info?

** Other options  
  1. Insert a module that 'masks' out data? How on EARTH would I jackknife by doing that, though? It buys me nothing.
  2. Create a 'held-out' data set which are understood by all modules, and use paramsets to jackknife
     - Confounds the splitter and jackknifing
     - Paramsets would have to be multiplied (2 paramsets x 10 jackknifes = 20 paramsets)
     - Would require splitters and unifiers be daisy-chained.
   
* THE PROBLEMS: 
** It's hard to add jackknifing and maintain simultaneous support for
     - multiple FIR filters
     - multiple paramsets
     - arbitrary performance metrics
     - fitting NPNL on the training data, and using it to predict the held-out data 
     - gui
** In other words
   1. If there are multiple parameter sets used to compute the held-out prediction, then anything you push into the model is ephemeral and a single recalc could wipe it out.
   2. It's hard to compute the 'correlation' without making this very specific to a particular model type.
   3. It's hard to do this without accidentally having the NPNL adjust its fit to the held-out data.


* How about this for an experiment?
  - Learn what a particular neuron does, what 'feature' it responds to
  - Remove that discrimination
  - Play the featureless sound back (Animal can't do the task) 
  - Play the featureless sound back with bogus stimulation (Animal can't do task)
  - Play the feauterless sound back with good stimulation (Animal can do the task)
  - Proof: that sound feature is represented by that part of the sound

* Decisions on Jackknifing that could be modified:
  - Instead of taking the mean, take the single best held-out predictor
    - Problem: doing that means you will always INCLUDE the outliers!
  - Maybe take the 'worst' held-out prediction?
* Found one problem:
  - Jackknifing is a dumb algorithm that just NANs out the data. This is simpler than playing with indexes and easier to reconstruct at the end.
  - However, when the FIR filter encounters NANs, it doesn't know what to do, and so the next NCOEFS-1 data points are also NAN
  - When the jackknifed data is brought back together again, we have N_COEFS-1 wide gaps between the points
  - This is probably the right thing to do (in general), in retrospect. Lucky it turned out that way. 
  - However, even better is to just NAN out the RESPAVG, not the stimulus. This requires some 'special knowledge' of the signal, but I'll do it for now.

   - NarfQuestions: For scientific queries
     | Field      | Type         | Null | Key | Default | Extra | Notes                                                |
     |------------+--------------+------+-----+---------+-------+------------------------------------------------------|
     | id         |              |      |     |         |       |                                                      |
     | name       | varchar(255) |      |     |         |       |                                                      |
     | status     | varchar(255) |      |     |         |       | When you have decided you have answered the question |
     | question   | text         |      |     |         |       | The question we asked.                               |
     | answer     | text         |      |     |         |       | Our current answer.                                  |
     | tags       | text         |      |     |         |       |                                                      |
     | modeltree  | text         |      |     |         |       | A model keyword tree; all branches are compared      |
     | batch      | varchar(255) |      |     |         |       | Associated with a particular batch                   |
     | summaryfig | text         |      |     |         |       | Summarizes the answer to the question                |
     | lastmod    | timestamp    |      |     |         |       |                                                      |

   - NarfBatches
     | batchname     | varchar(255) |   |   |   |   | More generic names                                   |
     | cellid        | varchar(255) |   |   |   |   |                                                      |
     | est_set       | text         |   |   |   |   |                                                      |
     | val_set       | text         |   |   |   |   |                                                      |
     | tags          | text         |   |   |   |   |                                                      |
     | lastmod       | timestamp    |   |   |   |   |                                                      |
** Data Structures
   - NarfResults
     | Field      | Type             | Null | Key | Default           | Extra                       | Notes                             |
     |------------+------------------+------+-----+-------------------+-----------------------------+-----------------------------------|
     | id         | int(10) unsigned | NO   | PRI | NULL              | auto_increment              |                                   |
     | cellid     | varchar(255)     | YES  | MUL | NULL              |                             |                                   |
     | batch      | int(11)          | YES  | MUL | NULL              |                             |                                   |
     | modelname  | text             | YES  | MUL | NULL              |                             |                                   |
     | r_test     | double           | YES  |     | NULL              |                             |                                   |
     | r_fit      | double           | YES  |     | NULL              |                             |                                   |
     | score      | double           | YES  |     | NULL              |                             |                                   |
     | sparsity   | double           | YES  |     | NULL              |                             |                                   |
     | modelpath  | text             | YES  |     | NULL              |                             |                                   |
     | modelfile  | text             | YES  |     | NULL              |                             |                                   |
     | figurefile | text             | YES  |     | NULL              |                             |                                   |
     | githash    | varchar(255)     | YES  |     | NULL              |                             |                                   |
     | lastmod    | timestamp        | NO   |     | CURRENT_TIMESTAMP | on update CURRENT_TIMESTAMP | When was the model last modified? |
     
     Add these:
     | Field           | Type      | Null | Key | Default | Extra | Notes                                                                            |
     |-----------------+-----------+------+-----+---------+-------+----------------------------------------------------------------------------------|
     | respfiles       | text      |      |     |         |       | contents of XXX0                                                                 |
     | lastVerified    | timestamp | YES  | ?   | ?       | ?     | When was the model/image/etc last checked for DB consistency?                    |
     | lastFit         | timestamp |      |     |         |       | When was the model last fit                                                      |
     | isHidden        | char(1)   | NO   |     |         |       | T or F                                                                           |
     | isDeleted       | char(1)   | NO   |     |         |       | T or F                                                                           |
     | estset          |           |      |     |         |       |                                                                                  |
     | valset          |           |      |     |         |       |                                                                                  |
     | est_l1          |           |      |     |         |       |                                                                                  |
     | est_mse         |           |      |     |         |       |                                                                                  |
     | est_corr        |           |      |     |         |       |                                                                                  |
     | est_likelihood  |           |      |     |         |       |                                                                                  |
     | est_aposteriori |           |      |     |         |       |                                                                                  |
     | est_bic         |           |      |     |         |       |                                                                                  |
     | jobStatus       |           |      |     |         |       | When created, set to 0. Shows Job system status so double-queuing doesn't occur. |
     | sparsity        | text      |      |     |         |       |                                                                                  |
     | smoothness      | text      |      |     |         |       |                                                                                  |
\section{TODO}


% If your fonts in generated saved figures aren't working properly in Kubuntu 12.04, do this:
sudo apt-get install xfonts-100dpi xfonts-75dpi
sudo add-apt-repository ppa:webupd8team/java
sudo apt-get update
sudo apt-get install oracle-java7-installer
% LOGOUT/LOGIN Required!
